\documentclass[a4paper, 9pt]{article}

% page layout
\usepackage{geometry}
\geometry{lmargin=1.3in, rmargin=1.3in}

% header formatting
\usepackage{sectsty}
\sectionfont{\large}

% danish letters and text symbols
\usepackage[utf8]{inputenc}
\usepackage[T1]{fontenc}

% links
\usepackage{hyperref}

% title page
\title{Emacs: C Guide}
\author{Alexander Christensen}

% emacs commands
\newcommand{\key}[1]{\texttt{"<#1>"}}


\begin{document}
\maketitle

\abstract{Guide til opsætning af C udviklingsmiljø i Emacs, herunder indentering, hurtig kompilering, auto-completion}

\section{Notation}
I Emacs jargon betyder \key{C-c} at man trykker "Control c" på sit tastatur, \key{M-c} betyder "Alt c", og \key{RET} betyder "enter".

\section{Intoduktion}
Denne guide henvender sig primært til studerende på DIKU, som skal til at påbegynde deres andet studieår, og som derfor skal lære at programmere \texttt{C}. Det forudantages, at læseren er bekendt med init-filen i sin Emacs-opsætning, hvor den er gemt, og hvordan den fungerer.\\

Nyttige kommandoer i denne sammenhæng: \key{C-e} i slutningen af en linje eller blok med E-lisp kode vil evaluere denne linje eller blok. \key{M-x} $\rightarrow$ "eval buffer" vil evaluere hele init-filen.


\newpage
\section{Indentering}
I en aktiv buffer, tast \key{C-c C-o}, og mini-bufferen vil vise den variabel, som definerer indenteringen på den pågældende linje. Tast \key{RET}, og du har mulighed for at ændre indenteringen (eller tast \key{C-g} for at afbryde).\\

Det er værd at bemærke, at denne indentering er afhængig af, om Emacs er indstillet til at indentere med tabs eller spaces! Til CompSys anbefales det at indentere med 2 spaces, og dette opnås ved at tilføje følende linjer til init-filen:
% TODO: bedre kode-formattering
\begin{verbatim}
(setq indent-tabs-mode nil)
(setq-default tab-width 2)
\end{verbatim}

\vspace{3mm}\noindent\textbf{\underline{eksempel:}}\vspace{2mm}\\
For at kunne indentere smertefrit med 2 spaces, kunne følgende linjer tilføjes til init-filen:
\begin{verbatim}
(defun my-c-mode-hook ()
  (setq-default c-basic-offset 2
                tab-width 2
                indent-tabs-mode nil)
  (c-set-offset 'defun-block-intro 2)
  (c-set-offset 'statement-block-intro 2)
  (c-set-offset 'comment-intro 0)
  (c-set-offset 'func-decl-cont 0)
  )

(add-hook 'c-mode-hook 'my-c-mode-hook)
\end{verbatim}\vspace{2mm}

\noindent
Hvis indentering 'driller', brug da metoden vist ovenfor: Undersøg, hvilken variabel, der definerer den pågældende linjes indentering, og tilføj rettelsen til init-filen.


\newpage
\section{Øvrige referencer}
\url{http://tuhdo.github.io/c-ide.html#sec-2}


\end{document}
